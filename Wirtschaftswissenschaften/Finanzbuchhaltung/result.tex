Auch: \textbf{Anfangsbestand}

\begin{quote}
{[}!DEFINITION{]} Der Anfangsbestand ist der Betrag, mit dem ein
{[}{[}Konto{]}{]} aus dem {[}{[}EBK{]}{]} zum Beginn der neuen
{[}{[}Rechnungsperiode{]}{]} quasi ``initialisiert'' wird.
\end{quote}

\begin{quote}
{[}!INFO{]} Der Anfangsbestand eines Kontos wird ueber das
{[}{[}EBK{]}{]} eingetragen. Bei einem {[}{[}Aktivkonto{]}{]} steht er
im {[}{[}Soll{]}{]} und bei einem {[}{[}Passivkonto{]}{]} im
{[}{[}Haben{]}{]}.
\end{quote}

\begin{quote}
{[}!SUMMARY{]} Aufgrund der {[}{[}Periodengerechte
Erfolgsrechnung\textbar periodengerechten Erfolgsrechnung{]}{]} wird der
Wertverlust eines Gegenstandes des
{[}{[}Anlagevermoegen\textbar Anlagevermoegens{]}{]} nicht ploetzlich,
sondern \textbf{ueber einen gewissen Zeitraum abgeschrieben.}
\end{quote}

\begin{quote}
{[}!INFO{]} Wenn ein
{[}{[}Anlagevermoegen\textbar Vermoegensgegenstand{]}{]} gekauft wird, der
abschreibungsfaehig ist, muss ein \textbf{Abschreibungsplan} fuer diesen
Gegenstand erstellt werden.

Wertminderungen dieses Gegenstandes werden durch planmaeßige oder
außerplanmaeßige Abschreibungen erfasst.
\end{quote}

\begin{quote}
{[}!IMPORTANT{]} \textbf{Wertgegenstaende bis 250€:} werden sofort
gebucht und nicht abgeschrieben.

\textbf{Wertgegenstaende zwischen 250€ und 1.000€:} Es wird ein
Sammelposten gebildet, der ueber 5 Jahre abgeschrieben wird.

\textbf{Wertgegenstaende zwischen 250€ und 800€:} Die
Vermoegensgegenstaende koennen sofort in voller Hoehe abgeschrieben werden.
(Die Entscheidung zwischen dieser und der vorherigen Option muss fuer das
ganze Jahr fuer alle Wirtschaftsgueter getroffen werden.)

\textbf{Wertgegenstaende ab 800€:} Die Vermoegensgegenstaende werden nach
der betriebsgewoehnlichen Nutzungsdauer anhand der AfA-Tabellen
abgeschrieben.
\end{quote}

\begin{quote}
{[}!Methoden{]} \textbf{Lineare Abschreibung} Die Abschreibung erfolgt
mit einem gleichbleibenden Prozentsatz von den
{[}{[}Anschaffungskosten{]}{]}. Dieser Prozentsatz laesst sich wie folgt
errechnen: (\textbf{Anschaffungskosten / Nutzungsdauer})

\textbf{Degressive Abschreibung} Diese Art der Abschreibung war
\textbf{nur von 2020 bis 2022} erlaubt! Hier ergibt sich der
Abschreibungsbetrag durch eine \textbf{Multiplizierung des Prozentsatzes
mit der Restsumme} des Vermoegensgegenstandes. Hier wird gewechselt ab
dem Moment, wo der Abschreibungswert der Linearen Abschreibung den der
Degressiven Abschreibung uebersteigt.
\end{quote}

\begin{quote}
{[}!QUOTE{]} \textbf{Unternehmen moegen Abschreibungen, weil sie das
{[}{[}Eigenkapital{]}{]} und den {[}{[}Gewinn{]}{]} mindern und damit
die Steuern senken.}
\end{quote}

\begin{quote}
{[}!QUOTE{]}- Laut {[}{[}HGB{]}{]}: \emph{Bei Vermoegensgegenstaenden des
{[}{[}Anlagevermoegen\textbar Anlagevermoegens{]}{]}, deren Nutzung
zeitlich begrenzt ist, sind die
{[}{[}Anschaffungskosten\textbar Anschaffungs{]}{]}- oder
{[}{[}Herstellungskosten{]}{]} um planmaeßige Abschreibungen zu
vermindern. Der Plan muss die Anschaffungs- oder Herstellungskosten auf
die {[}{[}Rechnungsperiode\textbar Geschaeftsjahre{]}{]} verteilen, in
denen der Vermoegensgegenstand voraussichtlich genutzt werden kann.
{[}\ldots{]} Ohne Ruecksicht darauf, ob ihre Nutzung zeitlich begrenzt
ist, sind bei Vermoegensgegenstaenden des Anlagevermoegens bei
voraussichtlich dauernder Wertminderung außerplanmaeßige Abschreibungen
vorzunehmen, um diese mit dem niedrigeren Wert anzusetzen, der ihnen am
Abschlussstichtag beizulegen ist. {[}\ldots{]}}
\end{quote}

\begin{quote}
{[}!DEFINITION{]} Eine
\textbf{{[}{[}Aktiv{]}{]}-{[}{[}Passiv{]}{]}-Mehrung} beschreibt eine
Wertaenderung zwischen mindestens einer {[}{[}Aktiv{]}{]}- und einer
{[}{[}Passiv{]}{]}-Position der {[}{[}Bilanz{]}{]}. Die
\textbf{{[}{[}Bilanzsumme{]}{]} erhoeht sich somit um den Wert der
{[}{[}Aktiv{]}{]}-{[}{[}Passiv{]}{]}-Mehrung}.
\end{quote}

\textbf{MERKE:} Die {[}{[}Aktiv{]}{]}-{[}{[}Passiv{]}{]}-Mehrung ist nur
einer von \textbf{vier Moeglichkeiten der
{[}{[}Bilanzveraenderung{]}{]}.}\textgreater{[}!DEFINITION{]}
\textgreater Eine
\textbf{{[}{[}Aktiv{]}{]}-{[}{[}Passiv{]}{]}-Minderung} beschreibt eine
Wertaenderung zwischen mindestens einer {[}{[}Aktiv{]}{]}- und einer
{[}{[}Passiv{]}{]}-Position der {[}{[}Bilanz{]}{]}. Die
\textbf{{[}{[}Bilanzsumme{]}{]} \emph{verringert} sich somit um den Wert
der {[}{[}Aktiv-Passiv-Mehrung{]}{]}}.

\textbf{MERKE:} Die {[}{[}Aktiv{]}{]}-{[}{[}Passiv{]}{]}-Minderung ist
nur einer von \textbf{vier Moeglichkeiten der
{[}{[}Bilanzveraenderung{]}{]}.}--- aliases: - Aktivseite - Aktiva ---
\#\#\# Allgemein \textbf{uebersicht:} AKTIVKONTO

\begin{longtable}[]{@{}ll@{}}
\toprule\noalign{}
Soll & Haben \\
\midrule\noalign{}
\endhead
\bottomrule\noalign{}
\endlastfoot
Anfangsbestand\textbf{Mehrungen} & \textbf{Minderungen}
Schlussbestand \\
\end{longtable}

\^{} Das {[}{[}Konto{]}{]} in der uebersicht Mehrungen stehen im Soll und
Minderungen im Haben.\textgreater{[}!DEFINITION{]} \textgreater Ein
Aktivtausch beschreibt eine Wertaenderung zwischen zwei Positionen auf
der \textbf{{[}{[}Aktiv{]}{]}seite} der {[}{[}Bilanz{]}{]}. Deren Summe
bleibt unveraendert.

\textbf{MERKE:} Der Aktivtausch ist nur einer von \textbf{vier
Moeglichkeiten der {[}{[}Bilanzveraenderung{]}{]}.}Auch:
\textbf{Anfangsbestand}

\begin{quote}
{[}!DEFINITION{]} Der Anfangsbestand ist der Betrag, mit dem ein
{[}{[}Konto{]}{]} aus dem {[}{[}EBK{]}{]} zum Beginn der neuen
{[}{[}Rechnungsperiode{]}{]} quasi ``initialisiert'' wird.
\end{quote}

\begin{quote}
{[}!INFO{]} Der Anfangsbestand eines Kontos wird ueber das
{[}{[}EBK{]}{]} eingetragen. Bei einem {[}{[}Aktivkonto{]}{]} steht er
im {[}{[}Soll{]}{]} und bei einem {[}{[}Passivkonto{]}{]} im
{[}{[}Haben{]}{]}.
\end{quote}

\begin{quote}
{[}!SUMMARY{]} Aufgrund der {[}{[}Periodengerechte
Erfolgsrechnung\textbar periodengerechten Erfolgsrechnung{]}{]} wird der
Wertverlust eines Gegenstandes des
{[}{[}Anlagevermoegen\textbar Anlagevermoegens{]}{]} nicht ploetzlich,
sondern \textbf{ueber einen gewissen Zeitraum abgeschrieben.}
\end{quote}

\begin{quote}
{[}!INFO{]} Wenn ein
{[}{[}Anlagevermoegen\textbar Vermoegensgegenstand{]}{]} gekauft wird, der
abschreibungsfaehig ist, muss ein \textbf{Abschreibungsplan} fuer diesen
Gegenstand erstellt werden.

Wertminderungen dieses Gegenstandes werden durch planmaeßige oder
außerplanmaeßige Abschreibungen erfasst.
\end{quote}

\begin{quote}
{[}!IMPORTANT{]} \textbf{Wertgegenstaende bis 250€:} werden sofort
gebucht und nicht abgeschrieben.

\textbf{Wertgegenstaende zwischen 250€ und 1.000€:} Es wird ein
Sammelposten gebildet, der ueber 5 Jahre abgeschrieben wird.

\textbf{Wertgegenstaende zwischen 250€ und 800€:} Die
Vermoegensgegenstaende koennen sofort in voller Hoehe abgeschrieben werden.
(Die Entscheidung zwischen dieser und der vorherigen Option muss fuer das
ganze Jahr fuer alle Wirtschaftsgueter getroffen werden.)

\textbf{Wertgegenstaende ab 800€:} Die Vermoegensgegenstaende werden nach
der betriebsgewoehnlichen Nutzungsdauer anhand der AfA-Tabellen
abgeschrieben.
\end{quote}

\begin{quote}
{[}!Methoden{]} \textbf{Lineare Abschreibung} Die Abschreibung erfolgt
mit einem gleichbleibenden Prozentsatz von den
{[}{[}Anschaffungskosten{]}{]}. Dieser Prozentsatz laesst sich wie folgt
errechnen: (\textbf{Anschaffungskosten / Nutzungsdauer})

\textbf{Degressive Abschreibung} Diese Art der Abschreibung war
\textbf{nur von 2020 bis 2022} erlaubt! Hier ergibt sich der
Abschreibungsbetrag durch eine \textbf{Multiplizierung des Prozentsatzes
mit der Restsumme} des Vermoegensgegenstandes. Hier wird gewechselt ab
dem Moment, wo der Abschreibungswert der Linearen Abschreibung den der
Degressiven Abschreibung uebersteigt.
\end{quote}

\begin{quote}
{[}!QUOTE{]} \textbf{Unternehmen moegen Abschreibungen, weil sie das
{[}{[}Eigenkapital{]}{]} und den {[}{[}Gewinn{]}{]} mindern und damit
die Steuern senken.}
\end{quote}

\begin{quote}
{[}!QUOTE{]}- Laut {[}{[}HGB{]}{]}: \emph{Bei Vermoegensgegenstaenden des
{[}{[}Anlagevermoegen\textbar Anlagevermoegens{]}{]}, deren Nutzung
zeitlich begrenzt ist, sind die
{[}{[}Anschaffungskosten\textbar Anschaffungs{]}{]}- oder
{[}{[}Herstellungskosten{]}{]} um planmaeßige Abschreibungen zu
vermindern. Der Plan muss die Anschaffungs- oder Herstellungskosten auf
die {[}{[}Rechnungsperiode\textbar Geschaeftsjahre{]}{]} verteilen, in
denen der Vermoegensgegenstand voraussichtlich genutzt werden kann.
{[}\ldots{]} Ohne Ruecksicht darauf, ob ihre Nutzung zeitlich begrenzt
ist, sind bei Vermoegensgegenstaenden des Anlagevermoegens bei
voraussichtlich dauernder Wertminderung außerplanmaeßige Abschreibungen
vorzunehmen, um diese mit dem niedrigeren Wert anzusetzen, der ihnen am
Abschlussstichtag beizulegen ist. {[}\ldots{]}}
\end{quote}

\begin{quote}
{[}!DEFINITION{]} Eine
\textbf{{[}{[}Aktiv{]}{]}-{[}{[}Passiv{]}{]}-Mehrung} beschreibt eine
Wertaenderung zwischen mindestens einer {[}{[}Aktiv{]}{]}- und einer
{[}{[}Passiv{]}{]}-Position der {[}{[}Bilanz{]}{]}. Die
\textbf{{[}{[}Bilanzsumme{]}{]} erhoeht sich somit um den Wert der
{[}{[}Aktiv{]}{]}-{[}{[}Passiv{]}{]}-Mehrung}.
\end{quote}

\textbf{MERKE:} Die {[}{[}Aktiv{]}{]}-{[}{[}Passiv{]}{]}-Mehrung ist nur
einer von \textbf{vier Moeglichkeiten der
{[}{[}Bilanzveraenderung{]}{]}.}\textgreater{[}!DEFINITION{]}
\textgreater Eine
\textbf{{[}{[}Aktiv{]}{]}-{[}{[}Passiv{]}{]}-Minderung} beschreibt eine
Wertaenderung zwischen mindestens einer {[}{[}Aktiv{]}{]}- und einer
{[}{[}Passiv{]}{]}-Position der {[}{[}Bilanz{]}{]}. Die
\textbf{{[}{[}Bilanzsumme{]}{]} \emph{verringert} sich somit um den Wert
der {[}{[}Aktiv-Passiv-Mehrung{]}{]}}.

\textbf{MERKE:} Die {[}{[}Aktiv{]}{]}-{[}{[}Passiv{]}{]}-Minderung ist
nur einer von \textbf{vier Moeglichkeiten der
{[}{[}Bilanzveraenderung{]}{]}.}--- aliases: - Aktivseite - Aktiva ---
\#\#\# Allgemein \textbf{uebersicht:} AKTIVKONTO

\begin{longtable}[]{@{}ll@{}}
\toprule\noalign{}
Soll & Haben \\
\midrule\noalign{}
\endhead
\bottomrule\noalign{}
\endlastfoot
Anfangsbestand\textbf{Mehrungen} & \textbf{Minderungen}
Schlussbestand \\
\end{longtable}

\^{} Das {[}{[}Konto{]}{]} in der uebersicht Mehrungen stehen im Soll und
Minderungen im Haben.\textgreater{[}!DEFINITION{]} \textgreater Ein
Aktivtausch beschreibt eine Wertaenderung zwischen zwei Positionen auf
der \textbf{{[}{[}Aktiv{]}{]}seite} der {[}{[}Bilanz{]}{]}. Deren Summe
bleibt unveraendert.

\textbf{MERKE:} Der Aktivtausch ist nur einer von \textbf{vier
Moeglichkeiten der {[}{[}Bilanzveraenderung{]}{]}.}Auch:
\textbf{Anfangsbestand}

\begin{quote}
{[}!DEFINITION{]} Der Anfangsbestand ist der Betrag, mit
de\textgreater{[}!QUOTE{]} Bei Kapitalgesellschaften und
Personengesellschaften mit Haftungsbeschraenkung bildet der Anhang den
dritten Pflichtbestandteil des
{[}{[}Jahresabschluss\textbar Jahresabschlusses{]}{]}. Der Anhang dient
vorrangig Informationszwecken und hat die Aufgabe, Erlaeuterungen zur
{[}{[}Bilanz{]}{]} sowie zur {[}{[}GUV\textbar Gewinn- und
Verlustrechnung{]}{]} vorzunehmen und sonstige Pflichtangaben zu
bestimmten Posten des Jahresabschlusses zu liefern:

\begin{enumerate}
\def\labelenumi{\arabic{enumi})}
\item
  Allgemeine Grundsaetze der Bilanzierung, Bewertung und
  Waehrungsumrechnung
\item
  Erlaeuternde, ergaenzende und korrigierende Informationen zur Bilanz und
  zur Gewinn- und Verlustrechnung, wie z.B. Wert und Entwicklung der
  einzelnen {[}{[}Anlagevermoegen\textbar Anlagegueter{]}{]}, Aufteilung
  der Umsaetze nach Taetigkeitsbereichen, \ldots{}
\item
  Sonstige Angaben, wie z.B. die Namen aller Mitglieder der
  Geschaeftsfuehrung und des Aufsichtsrates sowie Angaben ueber deren
  Bezuege und die Anzahl der beschaeftigten Arbeitnehmer. Aufbau und
  Gliederung des Anhangs sind im Gesetz nicht geregelt
\end{enumerate}
\end{quote}

\begin{quote}
{[}!SUMMARY{]} Das Anlagevermoegen beschreibt laut {[}{[}HGB{]}{]}
langfristige Anlagen, also Vermoegen oder Gegenstaende, die einem
Unternehmen langfristig zur Verfuegung stehen.
\end{quote}

Der Kaufpreis der Vermoegensgegenstaende bestimmt dabei die jeweiligen
{[}{[}Anschaffungskosten{]}{]}.

Bei einer Herstellung zaehlen die {[}{[}Herstellungskosten{]}{]}.
\textbf{Drei Hauptgruppen werden dabei unterschieden:}
\textgreater{[}!Immaterielles{]}-

\begin{quote}
{[}!Sachanlagen{]}-
\end{quote}

\subsection{\textgreater{[}!Finanzanlagen{]}-}\label{finanzanlagen-}

aliases: - Einkaufspreis --- \textgreater{[}!DEFINITION{]}
\textgreater Etwas anzuschaffen bedeutet, einen
{[}{[}Anlagevermoegen\textbar Vermoegensgegenstand{]}{]} fuer betriebliche
Zwecke zu kaufen.

\begin{quote}
{[}!WARNING{]} {[}{[}Anlagevermoegen\textbar Vermoegensgegenstaende{]}{]}
muessen zu Anschaffungskosten aktiviert werden.
\end{quote}

\begin{quote}
{[}!QUOTE{]}- Laut {[}{[}HGB{]}{]}: Anschaffungskosten sind die
Aufwendungen, die geleistet werden, um einen
{[}{[}Anlagevermoegen\textbar Vermoegensgegenstand{]}{]} zu erwerben und
ihn in einen betriebsbereiten Zustand zu versetzen, soweit sie dem
{[}{[}Anlagevermoegen\textbar Vermoegensgegenstand{]}{]} einzeln
zugeordnet werden koennen. Zu den Anschaffungskosten gehoeren auch die
Nebenkosten sowie die nachtraeglichen Anschaffungskosten.
Anschaffungspreisminderungen, die dem
{[}{[}Anlagevermoegen\textbar Vermoegensgegenstand{]}{]} einzeln
zugeordnet werden koennen, sind abzusetzen
\end{quote}

\begin{quote}
{[}!SUMMARY{]} Steht die Zahlung noch aus, aber der Erfolg ist bereits
in diesem Jahr passiert, wird ein Antizipativer RAP verwendet. Dieser
verschiebt die Kosten effektiv auf das korrekte Geschaeftsjahr.

Es gibt: \textbf{266 Sonstige Forderungen 489 Sonstige
Verbindlichkeiten}
\end{quote}

Wenn wir ``auf Ziel'' etwas kaufen oder jemand etwas von uns ``auf
Ziel'' kauft, entsteht eine {[}{[}Forderung{]}{]}. Auf Ziel kaufen
heißt, auf Rechnung kaufen, und das heißt final, dass die Summe erst
spaeter beglichen wird.

\begin{quote}
{[}!EXAMPLE{]} Kauft jemand bei einem Unternehmen etwas auf Ziel, hat
dieses Unternehmen der Person gegenueber eine
{[}{[}Forderung{]}{]}.\textgreater{[}!WARNING{]} Aufwandskonten sind
\textbf{Unterkonten vom {[}{[}Eigenkapital{]}{]}}. Sie haben
\textbf{keinen eigenen Anfangsbestand} ({[}{[}AB{]}{]}).
\end{quote}

In Ertragskonten werden {[}{[}Aufwendungen{]}{]} im {[}{[}Soll{]}{]},
also links, gebucht. (Siehe auch: {[}{[}Konto{]}{]}) Eine
\textbf{Aufwendung} bedeutet eine Ausgabe von Geld und somit eine
Verminderung des Erfolgs und des {[}{[}Eigenkapital{]}{]}s.Laut
{[}{[}HGB{]}{]} darf ein
\textbf{{[}{[}Geschaeftsvorgaenge\textbar Geschaeftsvorfall{]}{]}} (siehe
{[}{[}Geschaeftsvorgaenge{]}{]}) \textbf{nur dann gebucht werden, wenn ein
Beleg dafuer existiert.} Sollte es keinen Beleg geben, beispielsweise,
weil etwas gestohlen wurde oder kaputt gegangen ist, wird ein eigener
Beleg erstellt, um den Vorgang trotzdem buchen zu
koennen.\textgreater{[}!QUOTE{]} \textgreater Eine Bilanz ist die
gegliederte, ausgeglichene Gegenueberstellung des
{[}{[}Vermoegen\textbar Vermoegens{]}{]} und des Kapitals einer
Unternehmung, bewertet in Geldeinheiten. Auf der
{[}{[}Aktiv\textbar Aktivseite{]}{]} der Bilanz befinden sich das
{[}{[}Anlagevermoegen\textbar Anlage-{]}{]} sowie das
{[}{[}Umlaufvermoegen{]}{]}, auf der
{[}{[}Passiv\textbar Passivseite{]}{]} befinden sich Eigen- und
Fremdkapital.

Es existieren zwei Bilanzen. Zunaechst die \textbf{Eroeffnungsbilanz}, die
ueber das {[}{[}EBK{]}{]} gebildet wird und \textbf{im Endeffekt eine
Kopie der {[}{[}Schlussbilanz{]}{]}} ist und die
\textbf{{[}{[}Schlussbilanz{]}{]}}, in der die alle
{[}{[}Konto\textbar Konten{]}{]} ({[}{[}Konto{]}{]}) am Ende des Jahres
\textbf{ueber das {[}{[}SBK{]}{]} abgeschlossen} werden.

\begin{quote}
{[}!INFO{]} Die beiden Seiten der Bilanz heißen {[}{[}Aktiv{]}{]}a und
{[}{[}Passiv{]}{]}a. Diese beiden Seiten beinhalten jeweils
{[}{[}Konto\textbar Konten{]}{]}. Auf der rechten Seite, der
{[}{[}Passiv\textbar Passiva{]}{]}, stehen beispielsweise
{[}{[}Eigenkapital{]}{]} und {[}{[}Fremdkapital{]}{]}. Auf der linken
Seite steht {[}{[}Umlaufvermoegen{]}{]} oder auch feste Wertanlagen, wie
beispielsweise {[}{[}Rohstoffe{]}{]} oder auch
{[}{[}Wertpapieranlagen{]}{]}.
\end{quote}

\begin{quote}
{[}!INFO{]} Das Wort ``Bilanz'' kommt aus dem Italienischen, wo es fuer
\textbf{Waage} steht. Dies passt gut, da die \textbf{beiden Seiten einer
Bilanz immer ausgeglichen} sein muessen.
\end{quote}

\begin{quote}
{[}!QUOTE{]}- Nach {[}{[}HGB{]}{]}: Eine Bilanz wird immer zum
jeweiligen Stichtag, aber nicht am Stichtag erstellt. Verpflichtend sind
Bilanzen nach § 242 HGB bei Geschaeftseroeffnung, zum Ende eines jeden
Geschaeftsjahres sowie bei Aufloesung einer Unternehmung zu erstellen. Die
Rechnungsperiode erstreckt sich regelmaeßig auf einen Zeitraum von 12
Monaten (§ 240 Abs. 2 Satz 2 HGB).
\end{quote}

\begin{quote}
{[}!TIP{]} Desto \textbf{weiter unten} auf der
{[}{[}Aktiv\textbar Aktivseite{]}{]} einer Bilanz die Vermoegenswerte
stehen, \textbf{desto leichter sind sie liquide zu machen!} Daher stehen
beispielsweise Immobilien recht weit oben.
\end{quote}

Es gibt \textbf{vier Moeglichkeiten} der {[}{[}Bilanz{]}{]}aenderung: 1.
{[}{[}Aktivtausch{]}{]} 2. {[}{[}Passivtausch{]}{]} 3.
{[}{[}Aktiv-Passiv-Minderung{]}{]} 4.
{[}{[}Aktiv-Passiv-Mehrung{]}{]}--- aliases: - Buchungen ---
\textbf{Jeder {[}{[}Geschaeftsvorgaenge\textbar Geschaeftsvorgang{]}{]}}
(siehe {[}{[}Geschaeftsvorgaenge{]}{]}) \textbf{muss auf einem
{[}{[}Konto{]}{]} gebucht werden.} Dafuer wird je ein {[}{[}Beleg{]}{]}
benoetigt.

Buchungen werden neben dem Aufschrieb in den
{[}{[}Konto\textbar Konten{]}{]} als \textbf{{[}{[}Buchungssaetze{]}{]}}
festgehalten.

Fuer das {[}{[}System der doppelten Buchfuehrung{]}{]} ist wichtig, dass
\textbf{jeder Geschaeftsvorfall mit dem gleichen Betrag im
{[}{[}Soll{]}{]} und einmal im {[}{[}Haben{]}{]} gebucht} wird.

\begin{quote}
{[}!MERKE{]} Durch einen Geschaeftsvorfall werden die Bestaende von
mindestens zwei {[}{[}Konto\textbar Konten{]}{]} beruehrt. Daher sind
\textbf{{[}{[}Gegenbuchung{]}{]}en} erforderlich.
\end{quote}

\begin{quote}
{[}!IMPORTANT{]} Zuerst wird das {[}{[}Konto{]}{]} der Sollbuchung und
dann das der Habenbuchung aufgefuehrt.
\end{quote}

\begin{quote}
{[}!INFO{]} \textbf{Zusammengesetze Buchungssaetze} entstehen, wenn durch
einen Geschaeftsvorfall (Siehe auch: {[}{[}Geschaeftsvorgaenge{]}{]}) mehr
als zwei {[}{[}Konto\textbar Konten{]}{]} angesprochen werden.
\end{quote}

\begin{quote}
{[}!Example{]} Die Elektro GmbH kauft {[}{[}Rohstoffe{]}{]} im Wert von
EUR 4.500,- und zahlt EUR 500,- in bar und EUR 4.000,- per ueberweisung
Aus dem EBK werden die \textbf{{[}{[}AB{]}{]}s}, also die
\textbf{Anfangsbestaende auf die {[}{[}Konto\textbar Konten{]}{]}
gebucht.}
\end{quote}

\begin{quote}
{[}!QUOTE{]} Das Eroeffnungsbilanzkonto dient lediglich der Erfuellung
formeller Anforderungen des Systems der doppelten Buchfuehrung: Einer
Sollbuchung steht stets eine Habenbuchung in gleicher Hoehe gegenueber.
\end{quote}

(Siehe auch: {[}{[}System der doppelten
Buchfuehrung{]}{]},{[}{[}Konto{]}{]}) Das Eigenkapital eines Unternehmens
laesst sich berechnen, indem vom {[}{[}Vermoegen{]}{]} das
{[}{[}Fremdkapital{]}{]} abgezogen wird.

\begin{quote}
{[}!INFO{]} Eigenkapital ist ein {[}{[}Passivkonto{]}{]}
\end{quote}

Im Eigenkapital muessen sich {[}{[}Aufwendungen{]}{]} im {[}{[}Soll{]}{]}
und {[}{[}Ertraege{]}{]} im {[}{[}Haben{]}{]} zeigen. Da eine direkte
{[}{[}Buchung{]}{]} allerdings unuebersichtlich waere, wird zunaechst ueber
{[}{[}Erfolgskonten{]}{]} gebucht. Diese heißen auch
{[}{[}Aufwandskonten{]}{]} oder {[}{[}Ertragskonten{]}{]}.

\begin{quote}
{[}!Warning{]} \textbf{Das Eigenkapital sind die von den Eigentuemern des
Unternehmens bereitgestellten finanziellen Mittel. } (Siehe
{[}{[}Privatkonto{]}{]}) {[}!WARNING{]} \textbf{Erfolgskonten sind
Unterkonten des {[}{[}Eigenkapital{]}{]}s} Daher haben sie
\textbf{niemals eigene Anfangsbestaende!}
\end{quote}

{[}{[}Aufwendungen{]}{]} als Minderung des Eigenkapitals werden im Soll
der {[}{[}Aufwandskonten{]}{]} und {[}{[}Ertraege{]}{]} als Mehrung des
Eigenkapitals im Haben.

\begin{quote}
{[}!HINT{]} Die Salden aller Erfolgskonten werden schließlich ueber je
eine {[}{[}Gegenbuchung{]}{]} im {[}{[}GUV{]}{]}-{[}{[}Konto{]}{]}
gegengebucht.
\end{quote}

(Siehe auch: {[}{[}Konto{]}{]})\textgreater{[}!WARNING{]}
\textgreater Etragskonten sind \textbf{Unterkonten vom
{[}{[}Eigenkapital{]}{]}}. Sie haben \textbf{keinen eigenen
Anfangsbestand} ({[}{[}AB{]}{]}).

In Ertragskonten werden {[}{[}Ertraege{]}{]} im {[}{[}Haben{]}{]}, also
rechts, gebucht.

(Siehe auch: {[}{[}Konto{]}{]})Ein \textbf{Ertrag} ist ein Vorteil fuer
ein Unternehmen, da er den Erfolg mehrt und das {[}{[}Eigenkapital{]}{]}
erhoeht.

\begin{quote}
{[}!INFO{]} Spezifisch auf {[}{[}Buchung\textbar Buchungen{]}{]} bezogen
werden Ertraege im Haben der {[}{[}Ertragskonten{]}{]} gebucht.
\end{quote}

Eine Forderung zu haben, bedeutet, dass jemand Schulden einem gegenueber
hat.

Eine Forderung kann beispielsweise entstehen, wenn jemand etwas
{[}{[}auf Ziel{]}{]} kauft.

Es kann vorkommen, dass Forderungen einen Wertverlust hinnehmen, wenn
ein Schuldner seiner Zahlungsverpflichtung nicht nachkommen kann.
\textbf{Zum Bilanzstichtag werden Forderungen daher bewertet und ggf.
{[}{[}Abschreibungen\textbar abgeschrieben{]}{]}.}

\textbf{Forderungen werden aufgeteilt in drei Typen:}
\textgreater{[}!Uneinbringliche{]}- \textgreater Wenn eine Forderung
uneinbringlich geworden ist, wird sie unverzueglich in Hoehe ihres
Nettowertes {[}{[}Abschreibungen\textbar abgeschrieben{]}{]}.
\textgreater{}\textbf{Auch die Umsatzsteuer muss korrigiert werden!}
\textgreater Bsp: \textgreater{}\emph{695 {[}{[}Abschreibungen{]}{]} auf
Forderungen an 240 Forderungen aus LuL \textgreater480 Umsatzsteuer}

\begin{quote}
{[}!Zweifelhafte{]}- Wenn eine Forderung \textbf{zweifelhaft} geworden
ist, wird sie auf das Konto \textbf{zweifelhafte Forderungen} umgebucht.
Dort wird sie am {[}{[}Jahresabschluss{]}{]} mit einem neuen Wert
versehen, von dem der wahrscheinliche Zahlungsausfall abgezogen wird.

\textbf{Direkte Abschreibung:} Diese abgezogene Summe wird mit ihrem
\textbf{Nettobetrag} {[}{[}Abschreibungen\textbar abgeschrieben{]}{]}.

\textbf{Indirekte Abschreibung:} Hier wird die Abschreibung ueber
Einzelwertberichtigungen vorgenommen.

\textbf{Nach tatsaechlichem Zahlungsausfall:} Die Wertkorrektur muss
korrigiert werden und schließlich wird die Umsatzsteuer korrigiert.

\textbf{Zweifelhafte Forderungen sind brutto auszuweisen.} \textbf{Die
Umsatzsteuer darf erst korrigiert werden, wenn die Forderung geschlossen
ist!}
\end{quote}

\begin{quote}
{[}!Einwandfreie{]}- Das Fremdkapital eines Unternehmens ist Geld von
externen Geldgebern. Es laesst sich an der Bilanz ablesen oder auch durch
das Abziehen von {[}{[}Eigenkapital{]}{]} vom {[}{[}Vermoegen{]}{]}
berechnen.Eine Gegenbuchung muss aufgrund des \textbf{Systems der
doppelten Buchfuehrung} durchgefuehrt werden.
\end{quote}

\begin{quote}
{[}!Hint{]} Der Begriff ``Gegenbuchung'' sagt lediglich aus, dass jeder
Geschaeftsvorfall (siehe {[}{[}Geschaeftsvorgaenge{]}{]}) einmal im
{[}{[}Soll{]}{]} und einmal im {[}{[}Haben{]}{]} gebucht wird und daher
\textbf{zu jeder {[}{[}Buchung{]}{]} eine Gegenbuchung erforderlich}
ist.--- aliases: - Geschaeftsvorgang - Geschaeftsvorfall --- Auch:
\textbf{Geschaeftsvorfall}.
\end{quote}

\begin{quote}
{[}!Important{]} \textbf{Jeder Geschaeftsvorfall veraendert mindestens
zwei Positionen der {[}{[}Bilanz{]}{]}.}
\end{quote}

\textbf{Jeder Geschaeftsvorfall ist auf einem {[}{[}Konto{]}{]}
festzuhalten.} Dieser Vorgang nennt sich \textbf{{[}{[}Buchung{]}{]}}
und muss auf einem \textbf{entsprechenden {[}{[}Beleg{]}{]}}
beruhen.\textgreater{[}!Definition{]} \textgreater Ein \textbf{Gewinn}
liegt vor, wenn in einer {[}{[}Rechnungsperiode{]}{]} \textbf{mehr
{[}{[}Ertraege{]}{]} als {[}{[}Aufwendungen{]}{]}} vorliegen.

\textbf{{[}{[}KONTO{]}{]}NUMMER: \(802\)}

\begin{quote}
{[}!SUMMARY{]} Das \textbf{GUV steht fuer
Gewinn-und-Verlust-{[}{[}Konto{]}{]}}. In diesem {[}{[}Konto{]}{]}
werden alle {[}{[}Ertraege{]}{]} und {[}{[}Aufwendungen{]}{]} aus den
{[}{[}Erfolgskonten{]}{]} aufgefuehrt und schließlich ein
{[}{[}Gewinn{]}{]} oder ein {[}{[}Verlust{]}{]} fuer die jeweilige
{[}{[}Rechnungsperiode{]}{]} errechnet.
\end{quote}

\begin{quote}
{[}!QUOTE{]} Die Gewinn- und Verlustrechnung ist zeitraumbezogen, d.h.
sie erfasst die Aufwendungen und Ertraege einer
{[}{[}Rechnungsperiode\textbar Abrechnungsperiode{]}{]}.

Aus dem Saldo von {[}{[}Aufwendungen{]}{]} und
{[}{[}Ertraege\textbar Ertraegen{]}{]} ergibt sich entweder ▪ ein
{[}{[}Gewinn{]}{]}/Jahresueberschuss ({[}{[}Ertraege{]}{]} \textgreater{}
{[}{[}Aufwendungen{]}{]}) oder ▪ ein
{[}{[}Verlust{]}{]}/Jahresfehlbetrag ({[}{[}Aufwendungen{]}{]}
\textgreater{} {[}{[}Ertraege{]}{]}).

Ein Gewinn wuerde die Bilanzposition Eigenkapital erhoehen, ein Verlust
das Eigenkapital entsprechend reduzieren.
\end{quote}

\textbf{Zum {[}{[}Jahresabschluss{]}{]} wird das GuV im
{[}{[}Eigenkapital{]}{]} abgerechnet. Dafuer wird eine
{[}{[}Buchung{]}{]} vom GuV ans {[}{[}Eigenkapital{]}{]} erstellt.}

\begin{quote}
{[}!HINT{]} \textbf{{[}{[}Gewinn{]}{]} steht im GuV auf der
{[}{[}Soll{]}{]} und {[}{[}Verlust{]}{]} auf der
{[}{[}Haben{]}{]}-Seite.} Dadurch wird auch dieses {[}{[}Konto{]}{]}
ausgeglichen, da sonst die Seiten ja ungleich waeren.
\end{quote}

\begin{quote}
{[}!Example{]} \textbf{Gewinn- und Verlust-{[}{[}Konto{]}{]}}

\begin{longtable}[]{@{}ll@{}}
\toprule\noalign{}
Soll & Haben \\
\midrule\noalign{}
\endhead
\bottomrule\noalign{}
\endlastfoot
Loehne 10.000,- & Mietertraege 2.000,- \\
Bankzinsen 250,- & Zinsertraege 200,- \\
Rohstoffaufwendungen 4.000,- & {[}{[}Umsatzerloese{]}{]} 17.000,- \\
\textbf{Gewinn: 4.950,-} & / \\
19.200,- & 19.200,-\textgreater{[}!QUOTE{]} \\
\end{longtable}

\emph{Machen Sie sich keine Gedanken ueber Soll und Haben. Akzeptieren
Sie es einfach und denken Sie darueber als \textbf{Links und Rechts}.
\textasciitilde Frau Mumm, 2023}
\end{quote}

\textbf{{[}{[}SOLL{]}{]} = LINKS} \textbf{{[}{[}HABEN{]}{]} = RECHTS}

\begin{quote}
{[}!MERKE{]} Soll und Haben beschreiben die beiden Seiten eines
{[}{[}Konto{]}{]}s.\textgreater{[}!SUMMARY{]} Stellt ein Unternehmen
etwas selbst her, um es fuer betriebliche Zwecke zu nutzen, fallen
Herstellungskosten fuer diese ``Erschaffung'' an. Diese muessen gebucht
werden.
\end{quote}

Beispielsweise fallen {[}{[}Aufwendungen{]}{]} fuer Roh- oder
Betriebsstoffe oder Arbeitskraefte an.

\begin{quote}
{[}!WARNING{]} Bei einer Herstellung erhoeht sich der Wert der
Sachanlagen auf der {[}{[}Aktiv\textbar Aktivseite{]}{]} der
{[}{[}Bilanz{]}{]}.
\end{quote}

\textbf{Buchhalterische Erfassung:} Die {[}{[}Aufwendungen{]}{]} zur
Herstellung des Produktes werden neutralisiert, indem ein entsprechender
Wertzuwachs im \textbf{Konto 53 andere aktivierte Eigenleistungen}
gebucht wird.\textbf{HGB ist kurz fuer \emph{Handelsgesetzbuch}.} Es
bestimmt die grundlegenden Regeln fuer beispielsweise
{[}{[}Buchung\textbar Buchungen{]}{]}, aber auch, wie sich ein
{[}{[}Kaufmann{]}{]} zu verhalten hat.

Beispielsweise wird aufgelistet, dass jeder Kaufmann sowohl
{[}{[}Inventur{]}{]} zu machen, als auch eine {[}{[}Bilanz{]}{]} zu
erstellen hat. \textgreater{[}!Definition{]} \textgreater Das
\textbf{Inventar} ist die geordnete Liste des Bestands eines
Unternehmens, die aus der \textbf{{[}{[}Inventur{]}{]}} entsteht, die
laut {[}{[}HGB{]}{]} einmal im Jahr angefertigt werden muss.Die Inventur
muss laut {[}{[}HGB{]}{]} einmal im Jahr durchgefuehrt werden und
funktioniert im Sinne von \textbf{zaehlen, messen und wiegen}.

\begin{quote}
{[}!HINT{]} Die \textbf{Ergebnisse der Inventur} werden in einem
Bestandsverzeichnis, dem \textbf{{[}{[}Inventar{]}{]}} festgehalten.
\end{quote}

\begin{quote}
{[}!SUMMARY{]} Der Jahresabschluss wird am Bilanzstichtag durchgefuehrt.
Nach {[}{[}HGB{]}{]} beinhaltet er {[}{[}Bilanz{]}{]} und
{[}{[}GUV{]}{]}.
\end{quote}

\begin{quote}
{[}!INFO{]} Der Jahresabschluss wird immer \textbf{am letzten Tag des
laufenden Geschaeftsjahres} gemacht. Fuer gewoehnlich ist das der
\textbf{31.12.}
\end{quote}

\begin{quote}
{[}!HINT{]} Kapitalgesellschaften (bspw AGs oder GmbHs) haben zusaetzlich
zum normalen Jahresabschluss noch einen {[}{[}Anhang{]}{]} und außerdem
einen {[}{[}Lagebericht{]}{]} zu erstellen. \#\#\# Step by Step 1. Die
Schlussbilanz ist die Eroeffnungsbilanz. Zahlenmaeßig wird nichts
geaendert. 2. Die Anfangsbestaende werden ueber das {[}{[}EBK{]}{]} auf die
{[}{[}Konto\textbar Konten{]}{]} uebertragen. 3. Geschaeftsvorfaelle des
laufenden Jahres werden in die {[}{[}Konto\textbar Konten{]}{]} gebucht.
4. {[}{[}Konto\textbar Konten{]}{]} werden korrigiert, wenn Abweichungen
durch die Inventur auffallen. 5. Schlussbestaende werden im SBK
gebucht.\textgreater{[}!Info{]} Der Kontenplan ist die je nach
Unternehmen individuelle Zusammenstellung der
{[}{[}Konto\textbar Konten{]}{]}, die es benoetigt. Dabei werden die
jeweiligen {[}{[}Konto\textbar Konten{]}{]} aus dem gewaehlten
{[}{[}Kontenrahmen{]}{]} genommen und im Zweifel noch welche
hinzugefuegt, wenn notwendig. (Siehe {[}{[}Konto{]}{]})
\end{quote}

Je nach Branche werden unterschiedliche Kontenrahmen als Empfehlungen
angegeben, welche {[}{[}Konto\textbar Konten{]}{]} verwendet werden
koennten. (Siehe {[}{[}Konto{]}{]})

Um eine gewisse Einheitlichkeit zu schaffen, werden Kontonummern anstatt
von Namen fuer die {[}{[}Konto\textbar Konten{]}{]} verwendet. So wird
das Rechnungswesen rationalisierter.

Jedes Unternehmen sucht sich dabei diejenigen heraus, die fuer es
relevant sind. Daraus entsteht dann der {[}{[}Kontenplan{]}{]}.---
aliases: - Konten --- \textgreater{[}!INFO{]} \textgreater Ein Konto ist
kein im urspuenglichen Sinne gemeintes Konto, sondern kommt vom
italienischen Wort ``Conto'', was so viel wie \textbf{Rechung} bedeutet.

\begin{quote}
{[}!HINT{]} Ein Konto ist eine \emph{Mini-{[}{[}Bilanz{]}{]}}. Statt
\emph{{[}{[}Aktiv{]}{]}} und \emph{{[}{[}Passiv{]}{]}} heißen die Seiten
hier \emph{{[}{[}Soll{]}{]}} und \emph{{[}{[}Haben{]}{]}}. Auch auf
einem Konto muessen beide Seiten \textbf{immer ausgeglichen} sein.
\end{quote}

\paragraph{Arten von Konten}\label{arten-von-konten}

\begin{quote}
{[}!Bestandskonten{]} 1. \textbf{Aktivkonten} (siehe
{[}{[}Aktivkonto{]}{]}), also Konten, die auf der
{[}{[}Aktiv{]}{]}-Seite der {[}{[}Bilanz{]}{]} aufgefuehrt werden. 2.
\textbf{Passivkonten} (siehe {[}{[}Passivkonto{]}{]}), also Konten, die
auf der {[}{[}Passiv{]}{]}-Seite der {[}{[}Bilanz{]}{]} aufgefuehrt
werden.
\end{quote}

Konten muessen zu Beginn einer neuen {[}{[}Rechnungsperiode{]}{]} einmal
eroeffnet werden ({[}{[}Konteneroeffnung am Jahresbeginn{]}{]}). Dies
geschieht ueber das {[}{[}EBK{]}{]}.

\begin{quote}
{[}!QUOTE{]} Mittelgroße und große Kapitalgesellschaften sowie
entsprechende Personengesellschaften mit Haftungsbeschraenkung haben
neben dem aus Bilanz, Gewinn- und Verlustrechnung und Anhang bestehenden
Jahresabschluss einen Lagebericht zu erstellen. Die Erlaeuterungen des
Lageberichts sollen das Bild, welches durch den Jahresabschluss
entstanden ist, vervollstaendigen. Der Lagebericht hat alle Angaben zu
enthalten, die fuer die Gesamtbeurteilung des Geschaeftsverlaufs und des
Geschaeftsergebnisses sowie der wirtschaftlichen Lage, einschließlich der
kuenftigen Entwicklung und deren Chancen und Risiken, erforderlich sind.

\subsection{Bestandteil des Lageberichts ist u.a. ein Nachtragsbericht,
in dem auf Vorgaenge von besonderer Bedeutung, die nach dem Schluss des
Geschaeftsjahres eingetreten sind, eingegangen
wird.}\label{bestandteil-des-lageberichts-ist-u.a.-ein-nachtragsbericht-in-dem-auf-vorguxe4nge-von-besonderer-bedeutung-die-nach-dem-schluss-des-geschuxe4ftsjahres-eingetreten-sind-eingegangen-wird.}

aliases: - Passivseite - Passiva --- \#\#\# Allgemein
\textbf{uebersicht:} Passivkonto
\end{quote}

\begin{longtable}[]{@{}ll@{}}
\toprule\noalign{}
Soll & Haben \\
\midrule\noalign{}
\endhead
\bottomrule\noalign{}
\endlastfoot
\textbf{Minderungen} Schlussbestand &
Anfangsbestand\textbf{Mehrungen} \\
\end{longtable}

\^{} Das {[}{[}Konto{]}{]} in der uebersicht Minderungen stehen im Soll
und Mehrungen im Haben. \textgreater{[}!DEFINITION{]} \textgreater Ein
{[}{[}Aktivtausch{]}{]} beschreibt eine Wertaenderung zwischen zwei
Positionen auf der \textbf{{[}{[}Passiv{]}{]}seite} der
{[}{[}Bilanz{]}{]}. Deren Summe bleibt unveraendert.

\textbf{MERKE:} Der {[}{[}Aktivtausch{]}{]} ist nur einer von
\textbf{vier Moeglichkeiten der
{[}{[}Bilanzveraenderung{]}{]}.}\textgreater{[}!SUMMARY{]}
\textgreater Der Begriff beschreibt, dass
{[}{[}Buchung\textbar Buchungen{]}{]}, die erfolgstechnisch erst das
naechste Jahr betreffen, auch erst im naechsten Jahr erfolgswirksam werden
sollen. \textgreater Dafuer werden {[}{[}Buchung\textbar Buchungen{]}{]}
vorgenommen, um die Betraege quasi zu ``speichern''.

Fuer die Vorgaenge der Periodengerechten Erfolgsrechnung werden
{[}{[}Rechnungsabgrenzung\textbar Rechnungsabgrenzungsposten{]}{]}
genutzt.

\begin{quote}
{[}!INFO{]} Ein Preisnachlass ist eine nachtraegliche Korrektur des
Wertes der Lieferung oder der Leistung.
\end{quote}

Ein Preisnachlass wird ueber je ein eigenes {[}{[}Konto{]}{]}
abgeschlossen. \textbf{Beim Einkauf ueber das {[}{[}Konto{]}{]} 2002
Einstandspreiskorrektur (EPK)}

\textbf{Beim Verkauf ueber das {[}{[}Konto{]}{]} 5001 Erloeskorrektur}

\begin{quote}
{[}!WARNING{]} Bei Preisnachlaessen muss anteilig die
{[}{[}Vorsteuer{]}{]} oder {[}{[}Umsatzsteuer{]}{]} korrigiert werden.
Entscheidet sich einer der Eigentuemer des Unternehmens,~seine
Kapitaleinlage in der Unternehmen zu erhoehen oder steigt ein neuer
Unternehmer ein, \textbf{erhoeht sich das {[}{[}Eigenkapital{]}{]}} und
somit auch das {[}{[}Vermoegen{]}{]} des Unternehmens.
\end{quote}

=\textgreater{} Die Privatentnahme ist das Gegenteil
zur~{[}{[}Privateinlage{]}{]}. =\textgreater{} Fuer eine bessere
uebersichtlichkeit der Entnahmen und Einlagen wird das {[}{[}Konto{]}{]}
{[}{[}Privatkonto{]}{]} als Unterkonto des {[}{[}Eigenkapital{]}{]}s
eingerichtet.Entscheidet sich einer der Eigentuemer des Unternehmens,
\textbf{fuer seine privaten Zwecke Geld aus dem {[}{[}Eigenkapital{]}{]}
zu entnehmen}, spricht man von einer Privatentnahme. Diese mindert das
Eigenkapital.

Ein Unternehmer kann Geld als Vorschuss auf einen zu erwartenden
{[}{[}Gewinn{]}{]} oder als Minderung seiner bisherigen Kapitaleinlage
entnehmen.

=\textgreater{} Die Privatentnahme ist das Gegenteil zur
{[}{[}Privateinlage{]}{]}. =\textgreater{} Fuer eine bessere
uebersichtlichkeit der Entnahmen und Einlagen wird das {[}{[}Konto{]}{]}
{[}{[}Privatkonto{]}{]} als Unterkonto des {[}{[}Eigenkapital{]}{]}s
eingerichtet.\textgreater{[}!Warning{]} \textgreater Das Privatkonto ist
ein \textbf{Unterkonto vom {[}{[}Eigenkapital{]}{]}.} Es haelt
{[}{[}Privateinlage{]}{]}n und {[}{[}Privateinlage{]}{]}n fest.

\textbf{Einlagen} werden auf dem Privatkonto \textbf{im
{[}{[}Haben{]}{]}} und \textbf{Entnahmen im {[}{[}Soll{]}{]}} gebucht.
Zum {[}{[}Jahresabschluss{]}{]} wird das Saldo des Kontos auf diejenige
Seite geschrieben, die kleiner ist und dieses Saldo wird im Eigenkapital
ueber eine {[}{[}Buchung{]}{]} erfasst. \textgreater{[}!SUMMARY{]}
\textgreater Um der {[}{[}Periodengerechte
Erfolgsrechnung\textbar periodengerechten Erfolgsrechnung{]}{]} zu
folgen, werden Zahlungsvorgaenge, die nicht in diesem Jahr auch
erfolgswirksam sind, auf ihre entsprechende Zeitperiode verschoben.

Hierfuer wird zwischen {[}{[}Ausgabe{]}{]} und
{[}{[}Aufwendungen\textbar Aufwand{]}{]} sowie {[}{[}Einnahme{]}{]} und
tatsaechlichem {[}{[}Ertraege\textbar Ertrag{]}{]} unterschieden.

\begin{quote}
{[}!IMPORTANT{]} \textbf{Vorbereitende Jahresabschlussbuchung}

\begin{longtable}[]{@{}
  >{\raggedright\arraybackslash}p{(\linewidth - 6\tabcolsep) * \real{0.1765}}
  >{\raggedright\arraybackslash}p{(\linewidth - 6\tabcolsep) * \real{0.2941}}
  >{\raggedright\arraybackslash}p{(\linewidth - 6\tabcolsep) * \real{0.2941}}
  >{\raggedright\arraybackslash}p{(\linewidth - 6\tabcolsep) * \real{0.2353}}@{}}
\toprule\noalign{}
\begin{minipage}[b]{\linewidth}\raggedright
Fall
\end{minipage} & \begin{minipage}[b]{\linewidth}\raggedright
Geschaeftsvorfall im laufenden Jahr
\end{minipage} & \begin{minipage}[b]{\linewidth}\raggedright
Geschaeftsvorfall im folgenden Jahr
\end{minipage} & \begin{minipage}[b]{\linewidth}\raggedright
Abgrenzungsart
\end{minipage} \\
\midrule\noalign{}
\endhead
\bottomrule\noalign{}
\endlastfoot
1 & Ausgabe & Aufwand & Aktiver Rechnungsabgrenzungsposten (aktiver RAP)
\(293\) \\
2 & Einnahme & Ertrag & Passiver Rechnungsabgrenzungsposten (passiver
RAP) \(490\) \\
3 & Ertrag & Einnahme & Sonstige Forderungen \(266\) \\
4 & Aufwand & Ausgabe & Sonstige Verbindlichkeiten \(489\) \\
\end{longtable}
\end{quote}

Aktiver RAP und Passiver RAP sind {[}{[}Transitorische
Rechnungsabgrenzung\textbar Transitorische
Rechnungsabgrenzungsposten{]}{]}.

Sonstige Forderungen und sonstige Verbindlichkeiten sind
{[}{[}Antizipative Rechnungsabgrenzungsposten{]}{]}. Auch:
\textbf{Abrechnungsperiode}

\begin{quote}
{[}!Definition{]} Eine Rechnungsperiode ist der \textbf{Zeitraum
zwischen dem Start eines Geschaeftsjahres und dem
{[}{[}Jahresabschluss{]}{]}}.\textbf{{[}{[}KONTO{]}{]}NUMMER: \(600\)}
Werden {[}{[}Rohstoffe{]}{]} aus dem Lager in die Produktion gegeben,
wird dafuer eine eigene {[}{[}Buchung{]}{]} erstellt. Der Buchungssatz
dafuer ist: \emph{Rohstoffaufwendungen an Rohstoffe}
\end{quote}

Fuer die Ermittlung des Rohstoffverbrauches stehen drei Moeglichkeiten zur
Verfuegung: 1. Ermittlung durch Materialentnahmescheine (auch:
Skontraktionsmethode) 2. Ermittlung durch {[}{[}Inventur{]}{]} 3.
Ermittlung durch
{[}{[}Just-In-Time-Produktion{]}{]}\textgreater{[}!QUOTE{]}
\textgreater Im Falle von Ruecksendungen wird eine Annullierung bzw. eine
Korrektur der beim Einkauf bzw. beim Verkauf zuviel gebuchten Betraege
durch eine einfache Rueckbuchung bzw. Stornierung notwendig.

\begin{quote}
{[}!WARNING{]} Bei einer Stornierung einer Bestellung bzw einer
Rueckbuchung eines
{[}{[}Geschaeftsvorgaenge\textbar Geschaeftsvorfall{]}{]}es wird ebenfalls
eine \textbf{Verminderung der {[}{[}Vorsteuer{]}{]} oder
{[}{[}Umsatzsteuer{]}{]} notwendig}.
\end{quote}

Das \textbf{SBK ist das {[}{[}Schlussbilanz{]}{]}{[}{[}konto{]}{]}},
ueber das alle {[}{[}Geschaeftsvorgaenge{]}{]} abgerechnet werden, bevor
sie tatsaechlich auf die finale {[}{[}Bilanz{]}{]} uebertragen werden
koennen.

Da das SBK ein {[}{[}Konto{]}{]} ist, heißen seine beiden Seiten
{[}{[}Soll{]}{]} und {[}{[}Haben{]}{]}, anders als in der tatsaechlichen
Schlussbilanz.Die Schlussbilanz wird am Endes eines Jahres mithilfe der
{[}{[}GUV{]}{]} und des {[}{[}SBK{]}{]} erstellt. Sie ist die Bilanz,
die fuer den {[}{[}Jahresabschluss{]}{]} notwendig ist.

\begin{quote}
{[}!QUOTE{]} Wenn neben der Richtigkeit aller
{[}{[}Buchung\textbar Buchungen{]}{]} die erfassten
Bestandsveraenderungen den tatsaechlichen Veraenderungen entsprechen, muss
das Schlussbilanzkonto die gleichen Endbestaende aufweisen, wie eine zu
dem Zeitpunkt durchgefuehrte Inventur.
\end{quote}

\textbf{Die Seiten der Schlussbilanz heißen {[}{[}Aktiv{]}{]}a und
{[}{[}Passiv{]}{]}a}, anders als beim {[}{[}SBK{]}{]}.--- aliases: -
Skonto --- \textgreater{[}!INFO{]} \textgreater Skonto beschreibt einen
Preisnachlass als ``Belohnung'' fuer besonders fruehe Zahlungen.

Skonto stellt fuer den \textbf{Kaeufer eine Minderung des Kaufpreises} dar
und wird auf dem \textbf{Konto 2002 Einstandspreiskorrektur (EPK)}
gebucht.

\ldots{} fuer den \textbf{Verkaeufer eine Minderung des Verkaufserloeses}
dar und wird auf dem \textbf{Konto 5001 Erloeskorrektur} gebucht.

\begin{quote}
{[}!WARNING{]} Bei Skonti muss die {[}{[}Vorsteuer{]}{]} oder die
{[}{[}Umsatzsteuer{]}{]} anteilig korrigiert werden.
\end{quote}

\begin{quote}
{[}!WARNING{]} Ein auf einer Rechnung ausgewiesener Sofortrabatt hat
keinen buchhalterischen Bezug zu den {[}{[}Anschaffungskosten{]}{]}, da
lediglich der {[}{[}Anschaffungskosten\textbar Einkaufspreis{]}{]}
vermindert wird. Daher werden \textbf{Sofortrabatte buchhalterisch nicht
erfasst}.
\end{quote}

Sofortrabatte muessen von {[}{[}Ruecksendungen{]}{]},
{[}{[}Preisnachlass\textbar Preisnachlaessen{]}{]} oder
{[}{[}Skonti{]}{]} unterschieden werden.\textgreater{[}!QUOTE{]}
\textgreater{}\emph{Machen Sie sich keine Gedanken ueber Soll und Haben.
Akzeptieren Sie es einfach und denken Sie darueber als \textbf{Links und
Rechts}. \textgreater{} \textasciitilde Frau Mumm, 2023}

\textbf{{[}{[}SOLL{]}{]} = LINKS} \textbf{{[}{[}HABEN{]}{]} = RECHTS}

\begin{quote}
{[}!MERKE{]} Soll und Haben beschreiben die beiden Seiten eines
{[}{[}Konto{]}{]}s.\textgreater{[}!SUMMARY{]} Ist die Zahlung schon
geschehen, aber der dazugehoerige Erfolg gehoert ins naechste
{[}{[}Rechnungsperiode\textbar Geschaeftsjahr{]}{]}, wird der
transitorische Rechnungsabgrenzungsposten verwendet, um die
\textbf{Erfolgswirkung ebenfalls ins naechste Geschaeftsjahr zu buchen}.

Es gibt: \textbf{Aktiver RAP Passiver RAP}
\end{quote}

\begin{quote}
{[}!QUOTE{]}- Nach {[}{[}HGB{]}{]}: 1) Als Rechnungsabgrenzungsposten
sind auf der Aktivseite Ausgaben vor dem Abschlussstichtag auszuweisen,
soweit sie Aufwand fuer eine bestimmte Zeit nach diesem Tag darstellen.
2) Auf der Passivseite sind als Rechnungsabgrenzungsposten Einnahmen vor
dem Abschlussstichtag auszuweisen, soweit sie Ertrag fuer eine bestimmte
Zeit nach diesem Tag darstellen.
\end{quote}

\textbf{{[}{[}KONTO{]}{]}NUMMER: \(500\)} {[}{[}Erloese{]}{]} aus dem
Verkauf gefertigter Erzeugnisse werden auf dem {[}{[}Konto{]}{]}
\textbf{\(500\) Umsatzerloese} erfasst. Dieses \textbf{wird als
Ertragskonto} (siehe {[}{[}Ertragskonten{]}{]}) \textbf{ueber das
{[}{[}GUV{]}{]}-{[}{[}Konto{]}{]}} \(802\) abgeschlossen.
\textbf{{[}{[}KONTO{]}{]}NUMMER: \(480\)}
\textbf{{[}{[}Passivkonto{]}{]}}

\begin{quote}
{[}!DEFINITION{]} Der Wert, der durch Lieferungen und Leistungen
geschaffen wird, wird durch die Umsatzsteuer mit einem gewissen
Prozentsatz besteuert. AKA: Auf unsere {[}{[}Umsatzerloese{]}{]} muessen
wir Steuern zahlen.
\end{quote}

\begin{quote}
{[}!HINT{]} \textbf{In der Vorlesung wird mit 19\% gearbeitet.}
\end{quote}

\begin{quote}
{[}!QUOTE{]} Die Umsatzsteuer ist ein durchlaufender Posten ohne
Erfolgscharakter und langfristig ohne bestandsveraendernde Wirkung
\end{quote}

\begin{quote}
{[}!WARNING{]} \textbf{Durch die Umsatzsteuer entsteht eine
{[}{[}Verbindlichkeit{]}{]} dem Finanzamt gegenueber.}
\end{quote}

\textbf{MERKE: \emph{Die Umsatzsteuer faellt beim Verkauf von Produkten
oder Leistungen an!}}

ueber das {[}{[}Konto{]}{]} Vorsteuer wird die {[}{[}Vorsteuer{]}{]}
saldiert und ueber eine {[}{[}Gegenbuchung{]}{]} auf dem
{[}{[}Konto{]}{]} Umsatzsteuer erfasst. Nun zeigt es die Zahllast des
Betriebes. Dieser fuehrt die Umsatzsteuer an das Finanzamt ab.

\subsubsection{Abfuehrung der Steuer ans
Finanzamt}\label{abfuxfchrung-der-steuer-ans-finanzamt}

Da die Abfuehrung der Steuer erst 10 Tage nach Ablauf des
Voranmeldungszeitraumes erfolgt und zwischendurch eventuell ein neuer
Voranmeldungszeitraum beginnt, wird \textbf{ein Verrechnungskonto
erstellt}, auf dem \textbf{die Zahllast geparkt wird}. **Ist diese
Verrechnung nicht notwendig, werden die Salden der
{[}{[}Konto\textbar Konten{]}{]} {[}{[}Vorsteuer{]}{]} und
{[}{[}Umsatzsteuer{]}{]} einfach auf das {[}{[}Konto{]}{]} \emph{\(487\)
geleistete Vorsteuer / empfangene Umsatzsteuer} uebertragen.

Buchungssatz beim Einkauf: \emph{200 Rohstoffe \(208.845\),- an 440
Verb. aus LuL \(175.500\),- an 260 Vorsteuer \(33.345\),-} (Siehe auch:
{[}{[}Buchungssaetze{]}{]})Eine Verbindlichkeit entsteht, wenn jemand bei
uns eine {[}{[}Forderung{]}{]} hat, beispielsweise, weil wir etwas
{[}{[}Auf Ziel{]}{]} gekauft haben.Ein \textbf{Verlust} liegt vor, wenn
in der {[}{[}Rechnungsperiode{]}{]} \textbf{mehr
{[}{[}Aufwendungen{]}{]} als {[}{[}Ertraege{]}{]}}
vorliegen.\textbf{Vermoegen} ist im Endeffekt \textbf{direkt Kapital}.

Das Vermoegen eines Unternehmens wird berechnet, indem
\textbf{{[}{[}Eigenkapital{]}{]} und {[}{[}Fremdkapital{]}{]}} addiert
werden. \textbf{{[}{[}KONTO{]}{]}NUMMER: \(260\)}
\textbf{{[}{[}Aktivkonto{]}{]}} \textgreater{[}!INFO{]} \textgreater Die
Vorsteuer faellt beim {[}{[}Einkauf{]}{]} eines Produktes an. Sie ist
allerdings keine langfristige Schuld, weil wir sie an den Endverbraucher
weitergeben.

Der Betrag vom {[}{[}Konto{]}{]} Vorsteuer wird saldiert, dh es wird ein
Saldo gebildet, das wiederum ueber eine {[}{[}Gegenbuchung{]}{]} im
{[}{[}Konto{]}{]} {[}{[}Umsatzsteuer{]}{]} erfasst wird.

\begin{quote}
{[}!WARNING{]} Die Vorsteuer ist eine {[}{[}Forderung{]}{]}, die wir dem
Finanzamt gegenueber haben.
\end{quote}
